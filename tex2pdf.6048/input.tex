\documentclass[]{article}
\usepackage{lmodern}
\usepackage{amssymb,amsmath}
\usepackage{ifxetex,ifluatex}
\usepackage{fixltx2e} % provides \textsubscript
\ifnum 0\ifxetex 1\fi\ifluatex 1\fi=0 % if pdftex
  \usepackage[T1]{fontenc}
  \usepackage[utf8]{inputenc}
\else % if luatex or xelatex
  \ifxetex
    \usepackage{mathspec}
    \usepackage{xltxtra,xunicode}
  \else
    \usepackage{fontspec}
  \fi
  \defaultfontfeatures{Mapping=tex-text,Scale=MatchLowercase}
  \newcommand{\euro}{€}
\fi
% use upquote if available, for straight quotes in verbatim environments
\IfFileExists{upquote.sty}{\usepackage{upquote}}{}
% use microtype if available
\IfFileExists{microtype.sty}{%
\usepackage{microtype}
\UseMicrotypeSet[protrusion]{basicmath} % disable protrusion for tt fonts
}{}
\usepackage[margin=1in]{geometry}
\usepackage{longtable,booktabs}
\usepackage{graphicx}
\makeatletter
\def\maxwidth{\ifdim\Gin@nat@width>\linewidth\linewidth\else\Gin@nat@width\fi}
\def\maxheight{\ifdim\Gin@nat@height>\textheight\textheight\else\Gin@nat@height\fi}
\makeatother
% Scale images if necessary, so that they will not overflow the page
% margins by default, and it is still possible to overwrite the defaults
% using explicit options in \includegraphics[width, height, ...]{}
\setkeys{Gin}{width=\maxwidth,height=\maxheight,keepaspectratio}
\ifxetex
  \usepackage[setpagesize=false, % page size defined by xetex
              unicode=false, % unicode breaks when used with xetex
              xetex]{hyperref}
\else
  \usepackage[unicode=true]{hyperref}
\fi
\hypersetup{breaklinks=true,
            bookmarks=true,
            pdfauthor={Valentina Pucer},
            pdftitle={Poročilo pri predmetu Analiza podatkov s programom R},
            colorlinks=true,
            citecolor=blue,
            urlcolor=blue,
            linkcolor=magenta,
            pdfborder={0 0 0}}
\urlstyle{same}  % don't use monospace font for urls
\setlength{\parindent}{0pt}
\setlength{\parskip}{6pt plus 2pt minus 1pt}
\setlength{\emergencystretch}{3em}  % prevent overfull lines
\setcounter{secnumdepth}{0}

%%% Use protect on footnotes to avoid problems with footnotes in titles
\let\rmarkdownfootnote\footnote%
\def\footnote{\protect\rmarkdownfootnote}

%%% Change title format to be more compact
\usepackage{titling}

% Create subtitle command for use in maketitle
\newcommand{\subtitle}[1]{
  \posttitle{
    \begin{center}\large#1\end{center}
    }
}

\setlength{\droptitle}{-2em}
  \title{Poročilo pri predmetu Analiza podatkov s programom R}
  \pretitle{\vspace{\droptitle}\centering\huge}
  \posttitle{\par}
  \author{Valentina Pucer}
  \preauthor{\centering\large\emph}
  \postauthor{\par}
  \date{}
  \predate{}\postdate{}

\usepackage[slovene]{babel}
\usepackage{graphicx}


\begin{document}

\maketitle


\begin{verbatim}
## Warning: package 'dplyr' was built under R version 3.2.3
\end{verbatim}

\begin{verbatim}
## Warning: package 'rvest' was built under R version 3.2.3
\end{verbatim}

\begin{verbatim}
## Warning: package 'xml2' was built under R version 3.2.3
\end{verbatim}

\begin{verbatim}
## Warning: package 'gsubfn' was built under R version 3.2.3
\end{verbatim}

\begin{verbatim}
## Warning: package 'proto' was built under R version 3.2.3
\end{verbatim}

\begin{verbatim}
## Warning: package 'ggplot2' was built under R version 3.2.3
\end{verbatim}

\begin{verbatim}
## Warning: package 'plotrix' was built under R version 3.2.3
\end{verbatim}

\section{Izbira teme}\label{izbira-teme}

Za tematiko svoje seminarske naloge sem si izbrala analizo podatkov
naravnih nesreč v Združenih državah Amerike iz radovednosti, da izvem, v
kakšnem obsegu so naravne nesreče na drugi strani sveta, koli denarne
škode povzročijo..ipd.Naloge se se lotila tako, da sem podatke iz
spletne strani prenesla v program Microsoft Office Excel Worksheet in
oblikovala tabelo, ki se od originalne razlikuje v tem, da je nekaj
odvečnih podatkov odstranjenih, nekaj dodanih in urejenih. Drugo tabelo
sem uvozila iz spletne strani in jo tudi preuredila. Nanaša se na
podatke o številu smrtnih žrtev v dani naravni nesreči. S pomočjo teh
dveh tabel bom nato izdelala zemljevid.

\begin{center}\rule{0.5\linewidth}{\linethickness}\end{center}

\section{Obdelava, uvoz in čiščenje
podatkov}\label{obdelava-uvoz-in-ciscenje-podatkov}

Podatke sem uvozila v obliki .csv in direkt s spletne strani:

\begin{itemize}
\item
  ``\url{https://en.wikipedia.org/wiki/List_of_natural_disasters_in_the_United_States}''
\item
  ``\url{https://en.wikipedia.org/wiki/List_of_disasters_in_the_United_States_by_death_toll}''
\end{itemize}

Pri prvem uvozu sem tabeli uredila stolpce,tako da sem zamenjala njihov
vrstni red, odvečni zadnji stolpec prvotne tabele pobrisala. Pobrisala
sem tudi posamezne odvečne znake, ki so se pojavili v tabeli. Urediti
sem morala tudi števila npr. jih iz nizov dati v as.numeric. Pri
filtraciji podatkov sem glede na stolpec Škoda ocenila tip(stopnjo)
škode in temu dodala nov stolpec.

Spremenljivke pri prvi tabeli:

\begin{itemize}
\item
  ta dogodek(imenska spremenljivka)
\item
  lokacija(imenska spremenljivka \ldots{}.(za zemljevid))
\item
  leto, v katerem se je nesreča pripetila (številska spremenljivka)
\item
  tip naravne nesreče, ki se je pripetila (imenska spremenljivka)
\item
  število smrtnih žrtev(urejenostna smremenljivka (številska), HUDO, NI
  HUDO)
\item
  denarna škoda(urejenostna,\ldots{}.VELIKA ŠKODA, MANJŠA ŠKODA)
\end{itemize}

\begin{longtable}[c]{@{}lllllrl@{}}
\toprule
Leto & Lokacija & Tip.naravne.nesreče & Dogodek & Smrtne.žrtve & Škoda &
Stopnja\_skode\tabularnewline
\midrule
\endhead
2015 & Texas, Kansas, Oklahoma & Flood & 2015 Texas-Oklahoma floods & 46
& NA & NA\tabularnewline
2015 & Carolinas & Flood & October 2015 North American Storm Complex &
25 & 1e+09 & Manjša\tabularnewline
2015 & Utah & Flood & 2015 Utah floods & 20 & NA & NA\tabularnewline
2015 & Okanogan County, Washington & Wild fire & Okanogan Complex fire &
3 & NA & NA\tabularnewline
2014 & Oso, Washington & Mudflow & 2014 Oso mudslide & 43 & NA &
NA\tabularnewline
2014 & Nebraska, Louisiana,Oklahoma, Illinois, Florida,North Carolina &
Tornado & April 2014 tornado outbreak & 35 & 1e+09 &
Manjša\tabularnewline
\bottomrule
\end{longtable}

V drugem uvozu sem večidel stolpcev pobrisala. Ostala sta le dva prvotna
in tretji, ki se nanaša na število smrtnih žrtev pri neki naravni
nesreči(stopnja smrti oz.umrljivosti).Tudi tukaj sem morala številske
podatke prečistiti na ta način, da sem se znebila posameznih odvečnih
znakov in jih iz nizov pretvorila v številski podatek, s katerim sem
potem lahko upravljala.

\begin{longtable}[c]{@{}rrl@{}}
\toprule
Year & Fatalities & Stopnja\_smrti\tabularnewline
\midrule
\endhead
1900 & 6000 & Veliko\tabularnewline
1906 & 3000 & Veliko\tabularnewline
1928 & 3000 & Veliko\tabularnewline
2001 & 2996 & Veliko\tabularnewline
1941 & 2467 & Veliko\tabularnewline
1889 & 2209 & Veliko\tabularnewline
\bottomrule
\end{longtable}

Ideja za napoved podatkov naslednji fazi: Iz prve tabele odvisnost
ocenjene škode od lokacije naravne nesreče (razvitost območja) ali pa
vpliv tipa naravne nesreče na stopnjo denarje škode. Lahko bi sklepala,
da bodo posamezne manj razvite lokacije, kjer so se naravne nesreče
zgodile, imele več denarne škode, ker imajo manj sredstev za vlaganje v
zavarovanje pred temi dogodki (protipoplavni nasipi, protipotresne
zgradbe), s pomočjo katerih je škoda manjša. Še vedno pa imajo vpliv na
stopnjo denarne škode tudi drugi dejavniki, saj ne morem enačiti
hurikana s poplavo.

\begin{figure}

{\centering \includegraphics{projekt_files/figure-latex/torta-1} 

}

\caption{Tortni diagram}\label{fig:torta}
\end{figure}

\includegraphics{projekt_files/figure-latex/graf1-1.pdf} Dana tabela
prikazuje, kako seštevilo smrtni žrtev ki je večje od števila 600,
spreminja skozi leta. Opazimo lahko, da je število smrtnih žrtev v
19.stoletju približno 2000 na pripadajoče leto. Izjema je leto 1893,
kjer je 3000 smrtnih žrtev.

\begin{center}\rule{0.5\linewidth}{\linethickness}\end{center}

\end{document}
